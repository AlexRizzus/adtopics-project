\section{Analysis and results}
\label{sec:results}

\begin{table*}[t]
    \centering
    \begin{threeparttable}
    \begin{tabular}{||l|c|c|r||}
        \toprule
            Node type & Complexity & Memory cost & Communication cost \\
        \midrule \midrule
        Prover & $O(n^2log(n))$ & $160+10*c$ & \vtop{\hbox{\strut (send) $64 + 96 * h$ }\hbox{\strut (recv) $32*g + 94$ }} \\
    \midrule
        Aggregator & $O(n log(n))$ & $276+10*c+96*z$ & \vtop{\hbox{\strut (send) $(32*c+94)*nb + 64 + 96*z$}\hbox{\strut (recv) $(32*c+94) + 96* nb + 96*z$}} \\
    \midrule
        Verifier & $O(n^2log(n))$ & - & - \\
    \midrule
  \end{tabular}
  \begin{tablenotes}
      \item[1] n = number of bits in input
      \item[2] z = number of bad neighbours
      \item[3] c = number of counters
      \item[4] h = 1 if bad prover, 0 otherwise
      \item[5] g = number of good configurations  
    \end{tablenotes}
    \caption{Complexity, Memory and Communication costs for each node}
    \label{tab:1}
    \end{threeparttable}
\end{table*}


\subsection{Computational cost}
We now proceed to evaluate the computational, memory, and communication costs of this work's SANA implementation for Provers and Aggregators.
Starting from the computational cost, the most complex functions of the protocol are the sign and verify functions as they both have
to perform operations on the elliptic curve. These operations require several multiplications to be executed.
Table~\ref{tab:1} reports the estimations of the computational costs for the functions and the nodes using them. 
The variable \textbf{n} is the number of bits of the input given to the function, in this case, n = 256 bits.\\
It can safely be said that the highest cost falls on the Provers while the Aggregators, which only have to perform one multiplication between objects, face a lower computational load.
Memory costs for the Prover are contained as long as we choose a fair amount of counters. 
On the other hand, the aggregator's cost is linearly dependent on the number of bad neighbors, the node in fact has to memorize the public key and the message of each compromised node.\\

For what concerns the theoretical communication costs of the protocol, 
the aggregator has to send and receive more bytes depending on how many compromised devices 
it has as its neighbors.\\

After dealing with the theoretical costs it's crucial to shift to the practical test in order to estimate the true workload for the nodes.
We ran several simulations to investigate the time required to perform each action of the protocol (signature, aggregation, verify...) and have a proper estimation of the total time of execution of the algorithm.
For this calculation we used this formula:
\[(\sum_{i=0}^{n} p_i) * t_{agg} + d * (t_{ver} + t_{hash}) + t_{sign} \]
Where \textbf{n} is the number of the provers and \textbf{d} is the depth of the tree generated from the algorithm.
This provides a lower bound for the execution time of the whole protocol as it relies on the fact that all nodes 
can sign simultaneously and that at each level of the tree all the challenge verifications from the aggregators require the same time.
Unfortunately, this formula is not easy to use in real context as the devices move in space and the number of neighbors may change depending on the proximity of the other devices, so it's hard to assume what the depth of the tree can be and how much parallelization can happen while descending the tree.\\

Doing a comparison with the SANA's claims we obtained a higher communication cost due to our keys being 512 bits long instead of 256 and the software configuration of provers being hashed to 256 bits.\\
Instead, memory costs for provers are lower in our implementation, and in SANA's paper memory costs for the Aggregator are missing, so it's not possible to make a comparison.\\
Run-time costs are similar without considering the communication time not present in our simulation.\\

\subsection{Aggregation cost}
For what concerns the duration of the process of aggregation the observation focussed on one Aggregator aggregating the signature of up to 1000 provers.
Firstly, the aggregation time increases linearly with the number of Provers connected to the same Aggregator.
But this same aggregation time is not supposed to scale with the number of compromised Provers.\\
Figure \ref{fig:aggregation_comparison} demonstrates the previuos claims by showing no relevant discrepancy beetween the 0\% 25\% 50\% 75\% and 100\% bad provers percentage.

\subsection{Verification cost}
The other parameter tested was the verification time of the Aggregator. 
The parameter was the number of provers and the percentage of them not signing the default message.\\
Theoretically, the number of bad provers would make the verification time increase linearly, and on the other hand, the verification time should be independent of the number of legitimate provers.
This is due to the fact that the legitimate Provers sign the default message, thus being verified in constant time.\\
Figure \ref{fig:verification_comparison} confirms that the time of verification is in fact independent from the number of provers if they are all legitimate (0\% bad provers).
Instead, with 100\% of bad provers, the time for verification increase linearly with the number of provers, reaching up to 6000 seconds for the verification of 1000 bad provers.



\subsection{Memory cost}
In Figure \ref{fig:memory_cost} is represented the memory cost for the Aggregator.
Theoretically, this is expected to increase linearly with the number of bad neighbors. This is due to the nature of the protocol identifying the compromised devices and storing public keys and messages of all of them.\\
Figure \ref{fig:memory_cost} confirms this belief and shows how having more than just 10 bad provers makes the Aggregator use more than 1024Kb of memory.\\

\begin{figure*}
  \centering
  \includegraphics[width=.80\linewidth]{Images/aggregation.png}  
  \caption{Aggregation times comparison}
  \label{fig:aggregation_comparison}
\end{figure*}
\begin{figure*}
  \centering
  \includegraphics[width=.80\linewidth]{Images/verification.png}  
  \caption{Verification times comparison}
  \label{fig:verification_comparison}
\end{figure*}
\begin{figure*}
  \centering
  \includegraphics[width=.80\linewidth]{Images/memory_cost.png} 
  \caption{Memory cost for the aggregator} 
  \label{fig:memory_cost}
\end{figure*}
