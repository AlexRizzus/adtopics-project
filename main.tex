\documentclass[10pt, a4paper, twocolumn]{article} % 10pt font size (11 and 12 also possible), A4 paper (letterpaper for US letter) and two column layout (remove for one column)

\input{structure.tex} % Specifies the document structure and loads requires packages

%----------------------------------------------------------------------------------------
%	ARTICLE INFORMATION
%----------------------------------------------------------------------------------------

\title{On the security of collective \\ remote attestation protocols: the case study of SANA} % The article title

\author{
	\authorstyle{Alessandro Rizzo and Davide Maria Lazzaro} % Authors
	\newline\newline % Space before institutions
	\textsuperscript{1}\institution{Università degli studi di Padova, Padova, Italy}\\ % Institution 1
}

\date{\today}

%----------------------------------------------------------------------------------------

\begin{document}

\maketitle % Print the title

\thispagestyle{firstpage} % Apply the page style for the first page (no headers and footers)

%Se avete abstract, abilitate questo
\lettrineabstract{Software integrity for IoT devices is an open challenge in CyberSecurity. These devices have particular constraints in terms of memory and computational capabilities. SANA offers to solve this problem using an Optimistic Aggregated Signature scheme that lowers the computational and memory cost for the attestation. In this paper, we implement SANA using a modern ECC implementation called ''bn128'' (\cite{bn128_implementation}) in order to verify the claims of the original paper and review criticalities in the protocol. We show the performances of our implementation compared to the original ones and discuss if and how this protocol can be efficiently applied in the scenario of IoT devices.}


\section{Introduction}

The Internet of Things (IoT) paradigm promises to interconnect the world and
change the way data is collected and processed.
This revolution is largely built on the establishment of large networks of small 
devices that autonomously exchange data.
Typically, these devices have low computational capabilities and process
sensible data.
As a consequence, these networks attract the attention of malicious entities
that attempt to exploit the poor security standards usually employed in such
environments.

Remote Attestation (RA) has been suggested as a potential solution to mitigate the
security concerns.
It allows a remote entity (\ie Verifier) to validate the integrity of
the software of a local device (\ie Prover).
This process usually makes use of a signed hash value of the Prover’s software, 
generated by dedicated hardware.
To deal with scalability issues, Collective Remote Attestation (CRA) has 
been proposed as an adaptation of classical RA to check the status of a network 
in a decentralized fashion.

SANA(~\cite{sana}) is a CRA scheme that promises to inspect the overall
status of an IoT network with a single and efficient operation.
The attestation request is propagated by the owner to the best neighbor (\ie the node with the best connection) which proceeds to propagate the request to its neighbors.
This operation is repeated until all the interested provers are reached.
The provers, which are the leaves of the tree, respond by signing a message and then sending it back through the tree.
It does so by introducing a signature scheme called Optimistic Aggregated
Signature (OAS) allows the aggregation of the signatures coming from
different devices.
The aggregated signature is smaller than the naive one (\ie list of appended
signatures) and makes the verification cost depends only on the number of
compromised devices.
SANA comes with estimations about its memory, communication, and computational costs. But the formulas neglect to specify some variables that prove to be decisive to determine these estimations.
We chose to implement the protocol using an ECC algorithm based on Ethereum to handle the signing and verification phases of the protocol, then we analyzed the results to prove the claims that were given in the SANA paper and at last make some considerations about the criticalities of this protocol.

In this article we verify the security claims made in SANA's paper.
In particular, we focus on checking the overall cost for running the protocol
and the soundness of its propagation process.
We prove that the complexity of the verification is linear with the number of compromised devices.
On the other hand, we also discuss possible problems regarding the aggregation cost and memory cost for the Aggregators.
We discuss the criticalities in scenarios where one or more devices (\eg Aggregators, Provers) are manumitted or malfunctioning.
At last, we present some considerations on the feasibility of SANA in a scope where the devices are small, moving, and often unreliable.
\section{Premises}

\begin{figure*}
    \includegraphics[width=0.9\linewidth]{Images/SANA_general.png} % Figure image
    \caption{Schematic of SANA propagation tree} % Figure caption
\end{figure*}

\subsection{Elliptic curve cryptography}
Elliptic Curve Cryptography (ECC) is a public key
cryptographic method based on the algebraic structure of elliptic curves over
finite fields.
It was first presented by Neal Koblitz (\cite{ecc_kob}) and Victor S. Miller (\cite{ecc_mil}) in 1985 but algorithms based on Elliptic curves became of wide use only in 2004.
ECC security is based on the intractability of the elliptic
curve discrete logarithm problem.
In particular, it depends on the simplicity of computing a point multiplication 
on the curve as opposed to the intractability of computing the scalar multiplicand
given the original point and the result.

\subsection{SANA attestation protocol}
In SANA the remote attestation process starts from the verifier, which sends a request to the Owner that generates a verification Token. \\
The Verifier then creates a challenge from a random nonce and the token which then to the closest Aggregator in the propagation tree.
This node proceeds to propagate the challenge to its neighbors until the request reaches all the provers.
The latter sign a message based on the legitimacy of their software configurations. 
The signature travels back to the Aggregator which creates the aggregated signature of its neighbors. 
Lastly, the Verifier checks the aggregated signature of the whole tree and determines which devices
can be trusted and which are compromised.\\
For implementing this protocol we chose an elliptic-curve implementation used in Ethereum and called ''bn128'' as it uses a Barreto-Naehrig curve and it's said to offer 128 bits security.
In SANA, the ECC is used to generate public keys that belong to a multiplicative group composed of points on the elliptic curve of the system.
All the signatures are generated by a random oracle as points belonging to the curve. 
The verify function then uses an interesting property of the elliptic curves to perform a bilinear pairing of two points in the curve and check if the signature is correct.
The bilinear pairing is computed between points from different multiplicative groups but belonging to the same elliptic curve.
\section{Implementation details}


This SANA implementation uses ''bn128'', an ECC implementation that uses an already existing curve belonging to Barreto-Naherig set of curves.
This implementation uses the Barreto-Naehrig curve on a finite field Zp with p of 256 bits.
For the computation of the bilinear map ''bn128'' relies on the Tate-pairing technique which is the fastest for this task. In fact, it requires only one iteration of the Miller's loop function instead of three simplyfying a lot the computation of the pairing.
We chose this implementation because it's a moder versione of an Elliptic-curve cryptosystem used from Ethereum
It's used to generate keys, compute the signatures and verify them.
Our SANA implementation has been written and tested on python 3.7.
It defines a class for each actor in the protocol, they all inherit from the general Node class which contains all the common data structures and functions.
On top of that the classes define:
\begin{itemize}
    \item Owner: method handlerequest() to generate the Token and generateKeys() to generate keys for the new devices.
    \item Verifier: method storeToken() to receive and store the token in the tokenMem field.
    \item Aggregator: method aggregateSignature().
    \item Prover: getSoftConfig() to retrieve its hashed software configuration.
\end{itemize}
These classes are instantiated during the simulation and each object has a set of nodes assigned as neighbours in order to simulate a network of devices.
\section{Analysis and results}
\label{sec:results}

\begin{table*}[t]
    \centering
    \begin{threeparttable}
    \begin{tabular}{||l|c|c|r||}
        \toprule
            Node type & Complexity & Memory cost & Communication cost \\
        \midrule \midrule
        Prover & $O(n^2log(n))$ & $160+10*c$ & \vtop{\hbox{\strut (send) $64 + 96 * h$ }\hbox{\strut (recv) $32*g + 94$ }} \\
    \midrule
        Aggregator & $O(n log(n))$ & $276+10*c+96*z$ & \vtop{\hbox{\strut (send) $(32*c+94)*nb + 64 + 96*z$}\hbox{\strut (recv) $(32*c+94) + 96* nb + 96*z$}} \\
    \midrule
        Verifier & $O(n^2log(n))$ & - & - \\
    \midrule
  \end{tabular}
  \begin{tablenotes}
      \item[1] n = number of bits in input
      \item[2] z = number of bad neighbours
      \item[3] c = number of counters
      \item[4] h = 1 if bad prover, 0 otherwise
      \item[5] g = number of good configurations  
    \end{tablenotes}
    \caption{Complexity, Memory and Communication costs for each node}
    \label{tab:1}
    \end{threeparttable}
\end{table*}


\subsection{Computational cost}
We now proceed to evaluate the computational, memory, and communication costs of this work's SANA implementation for Provers and Aggregators.
Starting from the computational cost, the most complex functions of the protocol are the sign and verify functions as they both have
to perform operations on the elliptic curve. These operations require several multiplications to be executed.
Table~\ref{tab:1} reports the estimations of the computational costs for the functions and the nodes using them. 
The variable \textbf{n} is the number of bits of the input given to the function, in this case, n = 256 bits.\\
It can safely be said that the highest cost falls on the Provers while the Aggregators, which only have to perform one multiplication between objects, face a lower computational load.
Memory costs for the Prover are contained as long as we choose a fair amount of counters. 
On the other hand, the aggregator's cost is linearly dependent on the number of bad neighbors, the node in fact has to memorize the public key and the message of each compromised node.\\

For what concerns the theoretical communication costs of the protocol, 
the aggregator has to send and receive more bytes depending on how many compromised devices 
it has as its neighbors.\\

After dealing with the theoretical costs it's crucial to shift to the practical test in order to estimate the true workload for the nodes.
We ran several simulations to investigate the time required to perform each action of the protocol (signature, aggregation, verify...) and have a proper estimation of the total time of execution of the algorithm.
For this calculation we used this formula:
\[(\sum_{i=0}^{n} p_i) * t_{agg} + d * (t_{ver} + t_{hash}) + t_{sign} \]
Where \textbf{n} is the number of the provers and \textbf{d} is the depth of the tree generated from the algorithm.
This provides a lower bound for the execution time of the whole protocol as it relies on the fact that all nodes 
can sign simultaneously and that at each level of the tree all the challenge verifications from the aggregators require the same time.
Unfortunately, this formula is not easy to use in real context as the devices move in space and the number of neighbors may change depending on the proximity of the other devices, so it's hard to assume what the depth of the tree can be and how much parallelization can happen while descending the tree.\\

Doing a comparison with the SANA's claims we obtained a higher communication cost due to our keys being 512 bits long instead of 256 and the software configuration of provers being hashed to 256 bits.\\
Instead, memory costs for provers are lower in our implementation, and in SANA's paper memory costs for the Aggregator are missing, so it's not possible to make a comparison.\\
Run-time costs are similar without considering the communication time not present in our simulation.\\

\subsection{Aggregation cost}
For what concerns the duration of the process of aggregation the observation focussed on one Aggregator aggregating the signature of up to 1000 provers.
Firstly, the aggregation time increases linearly with the number of Provers connected to the same Aggregator.
But this same aggregation time is not supposed to scale with the number of compromised Provers.\\
Figure \ref{fig:aggregation_comparison} demonstrates the previuos claims by showing no relevant discrepancy beetween the 0\% 25\% 50\% 75\% and 100\% bad provers percentage.

\subsection{Verification cost}
The other parameter tested was the verification time of the Aggregator. 
The parameter was the number of provers and the percentage of them not signing the default message.\\
Theoretically, the number of bad provers would make the verification time increase linearly, and on the other hand, the verification time should be independent of the number of legitimate provers.
This is due to the fact that the legitimate Provers sign the default message, thus being verified in constant time.\\
Figure \ref{fig:verification_comparison} confirms that the time of verification is in fact independent from the number of provers if they are all legitimate (0\% bad provers).
Instead, with 100\% of bad provers, the time for verification increase linearly with the number of provers, reaching up to 6000 seconds for the verification of 1000 bad provers.



\subsection{Memory cost}
In Figure \ref{fig:memory_cost} is represented the memory cost for the Aggregator.
Theoretically, this is expected to increase linearly with the number of bad neighbors. This is due to the nature of the protocol identifying the compromised devices and storing public keys and messages of all of them.\\
Figure \ref{fig:memory_cost} confirms this belief and shows how having more than just 10 bad provers makes the Aggregator use more than 1024Kb of memory.\\

\begin{figure*}
  \centering
  \includegraphics[width=.80\linewidth]{Images/aggregation.png}  
  \caption{Aggregation times comparison}
  \label{fig:aggregation_comparison}
\end{figure*}
\begin{figure*}
  \centering
  \includegraphics[width=.80\linewidth]{Images/verification.png}  
  \caption{Verification times comparison}
  \label{fig:verification_comparison}
\end{figure*}
\begin{figure*}
  \centering
  \includegraphics[width=.80\linewidth]{Images/memory_cost.png} 
  \caption{Memory cost for the aggregator} 
  \label{fig:memory_cost}
\end{figure*}

\section{Conclusions}

Although many of the claims SANA made were verified in our implementation some other aspects were different and in particular situations this could lead to problems during the attestation process.
\subsection{Costs estimation}
While the computational and communication costs have proven to be similar the original paper omitted the memory cost for the Aggregators. 
The memory occupation for this actor grows linearly with the number of bad provers in the sub-tree of the node. 
When we think of a network with hundreds of thousands of devices it's easy to find some compromised devices and in particular for the Aggregators that are close to the Verifier in the tree this could be a problem.
Concluding it's hard to think that this protocol could be used without knowing that at least the closest Aggregators have a reasonable amount of memory.

\subsection{Concerns on the Aggregator role}
Since the typical use case for this protocol is a network of wireless and mobile devices, it's reasonable to assume that there will be eventually interferences and disconnections during the attestation.
The concerns are that if an Aggregator doesn't receive the signature of a node it cannot notify the Verifier about this. Thus the final verification will fail unless the Aggregator has a proper way to inform the Verifier about the missing signature.
This solution however would bring more overhead for the Aggregators.
Another concern is that a temporary disconnection of an Aggregator from the network would generate the same problem as above, leading to a failure of the final verification.
In general the main issue is that there is no way to recover lost signature or notify if some are missing. Thus making the protocol sensitive to jamming and DDoS attacks on the Aggregators. 

\subsection{Concerns on the Prover role}
When generating a response to the challenge, the Prover checks by itself if its software configuration is legitimate.
Therefore if an attacker is able to compromise the software executed from the Prover could pass the attestation without being noticed as an untrusted device.


\balance
\printbibliography[title={Bibliography}] % Print the bibliography, section title in curly brackets
\balance


\end{document}
